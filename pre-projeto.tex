%% abtex2-modelo-relatorio-tecnico.tex, v<VERSION> laurocesar
%% Copyright 2012-2015 by abnTeX2 group at http://www.abntex.net.br/ 
%%
%% This work may be distributed and/or modified under the
%% conditions of the LaTeX Project Public License, either version 1.3
%% of this license or (at your option) any later version.
%% The latest version of this license is in
%%   http://www.latex-project.org/lppl.txt
%% and version 1.3 or later is part of all distributions of LaTeX
%% version 2005/12/01 or later.
%%
%% This work has the LPPL maintenance status `maintained'.
%% 
%% The Current Maintainer of this work is the abnTeX2 team, led
%% by Lauro César Araujo. Further information are available on 
%% http://www.abntex.net.br/
%%
%% This work consists of the files abntex2-modelo-relatorio-tecnico.tex,
%% abntex2-modelo-include-comandos and abntex2-modelo-references.bib
%%

% ------------------------------------------------------------------------
% ------------------------------------------------------------------------
% abnTeX2: Modelo de Relatório Técnico/Acadêmico em conformidade com 
% ABNT NBR 10719:2011 Informação e documentação - Relatório técnico e/ou
% científico - Apresentação
% Adaptado por Daniel Saad Nogueira Nunes para uso no IFB Taguatinga.
% ------------------------------------------------------------------------ 
% ------------------------------------------------------------------------


% Como opção escolha 'bacharelado' ou 'licenciatura'
\documentclass[licenciatura]{pre-projeto-computacao}
% ---
% Informações de dados para CAPA e FOLHA DE ROSTO
% ---
\titulo{Título do Pré-projeto}
\autor{Virgulino Ferreira da Silva}
% para feminino use \orientadora{Nome da orientadora}
%\orientadora{Maria Bonita}
\orientador{Padre Cícero}

% para masculino use \coorientador{Nome do coorientador}
%\coorientador{Padre Cícero}
\coorientadora{Maria Bonita}

% Linha de pesquisa conforme tabela de áreas do CNPq
\areadepesquisa{Computabilidade e Modelos de Computação }
\local{Brasília, DF}
\data{2020}

\begin{document}

\selectlanguage{brazil}
\frenchspacing 
\imprimircapa
\imprimirfolhaderosto



\section*{Contextualização}
	Este modelo visa adaptar a classe ABNTeX2 para utilização no IFB Taguatinga como documento de Pré-projeto, o qual deverá ser submetido para matrícula na disciplina de Projeto de Conclusão de Curso.
	
	Neste modelo, apresentamos apenas uma \textbf{sugestão} de seções adotadas com as suas breves descrições. O aluno e orientador podem adotar outra divisão se acharem conveniente.
	
	A bibliografia deve seguir o padrão estabelecido pela classe ABNTeX2 \cite{abntex2modelo-relatorio}.
	
	A contextualização deverá situar o tema de estudo e introduzir o problema a ser tratado.

\section*{Proposta}
	A proposta do trabalho deve ser detalhada nesta seção.

\section*{Justificativa}
	Deve ser dada a devida menção da importância da pesquisa sobre o tema adotado.	

\section*{Objetivos}
	Os objetivos gerais devem ser elucidados nesta seção.

\bibliography{bibliografia}

\end{document}
